\newcommand{\hiddensubsubsection}[1]{
\stepcounter{subsubsection}
\subsubsection*{\arabic{section}.\arabic{subsection}.\arabic{subsubsection}\hspace{1em}{#1}}}

\section{Python Wrappers\label{pywrappers}}

The Python foundations of the \PSIfour driver and Psithon syntax permit
many commonly performed post-processing procedures to be integrated into
the \PSIfour suite.
Among these are automated computations of interaction energies through
\texttt{cp()}, of a model chemistry applied to a database of systems through
\texttt{db()}, and of several model chemistries together approximating greater
accuracy through \texttt{cbs()}, all of which are discussed below.
Note that the options documented for these wrappers are placed as arguments
in the command that calls the wrapper (formatted in this manual as
\pyoptionname{option = } \pyoptionval{value}),
not in the \texttt{set globals} block or with any other \texttt{set} command.

%\subsection{\texttt{counterpoise\_correct}\label{wrapcp}}
%\subsection{\texttt{database}\label{wrapdb}}
\subsection{complete\_basis\_set\label{wrapcbs}}

The \texttt{complete\_basis\_set()} or \texttt{cbs()} wrapper defines a
multistage energy method from combinations of basis set extrapolations
and delta corrections and condenses the components into a minimum
number of calculations.

As represented in the equation below, a CBS energy method is defined in
four sequential stages (scf, corl, delta, delta2) covering treatment of
the reference total energy, the correlation energy, a delta
correction to the correlation energy, and a second delta correction.
Each is activated by a stage\_wfn keyword and is only allowed if all
preceding stages are already active.

At present, the \optionname{BASIS} keyword in the global and
local options sections are not considered by this wrapper.

%\begin{align*}
%E_{total}^{\text{CBS}} = \;
%    & \mathcal{F}_{\pyoptionname{SCF\_SCHEME}} \left(E_{total,\; \text{SCF}}^{\pyoptionname{SCF\_BASIS}}\right) \; + \\
%    & \mathcal{F}_{\pyoptionname{CORL\_SCHEME}} \left(E_{corl,\; \pyoptionname{CORL\_WFN}}^{\pyoptionname{CORL\_BASIS}}\right) \; + \\
%    & \left. \delta_{\pyoptionname{DELTA\_WFN\_LESSER}}^{\pyoptionname{DELTA\_WFN}} \; +
%        \qquad \right \} \mathcal{F}_{\pyoptionname{DELTA\_SCHEME}} \left(E_{corl,\; \pyoptionname{DELTA\_WFN}}^{\pyoptionname{DELTA\_BASIS}}\right) -
%        \mathcal{F}_{\pyoptionname{DELTA\_SCHEME}} \left(E_{corl,\; \pyoptionname{DELTA\_WFN\_LESSER}}^{\pyoptionname{DELTA\_BASIS}}\right) \\
%    & \left. \delta_{\pyoptionname{DELTA2\_WFN\_LESSER}}^{\pyoptionname{DELTA2\_WFN}}
%        \qquad \right \} \mathcal{F}_{\pyoptionname{DELTA2\_SCHEME}} \left(E_{corl,\; \pyoptionname{DELTA2\_WFN}}^{\pyoptionname{DELTA2\_BASIS}}\right) -
%        \mathcal{F}_{\pyoptionname{DELTA2\_SCHEME}} \left(E_{corl,\; \pyoptionname{DELTA2\_WFN\_LESSER}}^{\pyoptionname{DELTA2\_BASIS}}\right)
%\end{align*}

\begin{equation*}
E_{total}^{\text{CBS}} = \mathcal{F}_{\pyoptionname{SCF\_SCHEME}} \left(E_{total,\; \text{SCF}}^{\pyoptionname{SCF\_BASIS}}\right) \; +
    \mathcal{F}_{\pyoptionname{CORL\_SCHEME}} \left(E_{corl,\; \pyoptionname{CORL\_WFN}}^{\pyoptionname{CORL\_BASIS}}\right) \; +
    \delta_{\pyoptionname{DELTA\_WFN\_LESSER}}^{\pyoptionname{DELTA\_WFN}} \; +
    \delta_{\pyoptionname{DELTA2\_WFN\_LESSER}}^{\pyoptionname{DELTA2\_WFN}}
\end{equation*}
Here, $\mathcal{F}$ is an energy or energy extrapolation scheme, and the following also hold.
\begin{equation*}
    \delta_{\pyoptionname{DELTA\_WFN\_LESSER}}^{\pyoptionname{DELTA\_WFN}} \; =
        \mathcal{F}_{\pyoptionname{DELTA\_SCHEME}} \left(E_{corl,\; \pyoptionname{DELTA\_WFN}}^{\pyoptionname{DELTA\_BASIS}}\right) -
        \mathcal{F}_{\pyoptionname{DELTA\_SCHEME}} \left(E_{corl,\; \pyoptionname{DELTA\_WFN\_LESSER}}^{\pyoptionname{DELTA\_BASIS}}\right)
\end{equation*}
\begin{equation*}
    \delta_{\pyoptionname{DELTA2\_WFN\_LESSER}}^{\pyoptionname{DELTA2\_WFN}} \; =
        \mathcal{F}_{\pyoptionname{DELTA2\_SCHEME}} \left(E_{corl,\; \pyoptionname{DELTA2\_WFN}}^{\pyoptionname{DELTA2\_BASIS}}\right) -
        \mathcal{F}_{\pyoptionname{DELTA2\_SCHEME}} \left(E_{corl,\; \pyoptionname{DELTA2\_WFN\_LESSER}}^{\pyoptionname{DELTA2\_BASIS}}\right)
\end{equation*}

\hiddensubsubsection{Required Arguments}

All wrappers require an (optionally) unnamed first argument \pyoptionname{name}
that specifies the quantum chemical method. This isn't exactly suitable for \texttt{cbs()},
which deals with multiple levels of theory through the \pyoptionname{stage\_wfn} keywords
below, so this first argument can be used in place of \pyoptionname{CORL\_WFN} for
correlated calculations or with value \pyoptionval{\qq{scf}\qq} otherwise. Specifying
\pyoptionname{CORL\_WFN} outright renders the first argument defunct. To run the
\texttt{cbs()} wrapper, at a minimum, the SCF basis (for a SCF wavefunction) or the wavefunction
and basis (for a correlated wavefunction) must be specified.
The following keyword combinations will launch some basic calculations.
\begin{Snippet}
cbs(name='mp2', corl_basis='cc-pVDZ')
cbs('scf', scf_basis='cc-pVDZ')
cbs('mp2', corl_basis='cc-pVDZ')
cbs('mp2', corl_basis='cc-pV[DT]Z', corl_scheme=corl_xtpl_helgaker_2)
cbs('scf', scf_basis='cc-pV[DTQ]Z', scf_scheme=scf_xtpl_helgaker_3)
\end{Snippet}

\hiddensubsubsection{Optional Arguments}

Default values for keywords are indicated by italics.

\begin{itemize}
\item Energy Methods

The presence of a \pyoptionname{stage\_wfn} keyword is the indicator to incorporate (and check for
\pyoptionname{stage\_basis} and \pyoptionname{stage\_scheme} keywords) and compute that stage in defining the CBS energy.
\begin{itemize}
\item[] \pyoptionname{CORL\_WFN} = \pyoptionval{\qq{mp2}\qq} \textbar\; \pyoptionval{\qq{ccsd(t)\qq}} \textbar\; etc. \\
Indicates the energy method for which the correlation energy is to be obtained. Can also be specified with \pyoptionname{name}
or as the unlabeled first argument to the wrapper.
\item[] \pyoptionname{DELTA\_WFN} = \pyoptionval{\qq{ccsd}\qq} \textbar\; \pyoptionval{\qq{ccsd(t)}\qq} \textbar\; etc. \\
Indicates the (superior) energy method for which a delta correction to the correlation energy is to be obtained.
\item[] \pyoptionname{DELTA\_WFN\_LESSER} = \textit{\pyoptionval{\qq{mp2}\qq}} \textbar\; \pyoptionval{\qq{ccsd}\qq} \textbar\; etc. \\
Indicates the inferior energy method for which a delta correction to the correlation energy is to be obtained.
\item[] \pyoptionname{DELTA2\_WFN} = \pyoptionval{\qq{ccsd}\qq} \textbar\; \pyoptionval{\qq{ccsd(t)\qq}} \textbar\; etc. \\
Indicates the (superior) energy method for which a second delta correction to the correlation energy is to be obtained.
\item[] \pyoptionname{DELTA2\_WFN\_LESSER} = \textit{\pyoptionval{\qq{mp2}\qq}} \textbar\; \pyoptionval{\qq{ccsd}\qq} \textbar\; etc. \\
Indicates the inferior energy method for which a second delta correction to the correlation energy is to be obtained.
\end{itemize}

\item Basis Sets

Currently, the basis sets set in set globals have no influence in a \texttt{cbs()} calculation.
\begin{itemize}
\item[] \textbf{\pyoptionname{SCF\_BASIS}} = \textit{\pyoptionname{CORL\_BASIS}} \textbar\; \pyoptionval{\qq{cc-pV[TQ]Z}\qq} \textbar\; \pyoptionval{\qq{jun-cc-pv[tq5]z}\qq} \textbar\; \pyoptionval{\qq{6-31G}\qq} \textbar\; etc. \\
Indicates the sequence of basis sets employed for the reference energy. If any correlation method is specified, \pyoptionname{scf\_basis} can default to \pyoptionname{corl\_basis}.
\item[] \pyoptionname{CORL\_BASIS} = \pyoptionval{\qq{cc-pV[TQ]Z}\qq} \textbar\; \pyoptionval{\qq{jun-cc-pv[tq5]z}\qq} \textbar\; \pyoptionval{\qq{6-31G}\qq} \textbar\; etc. \\
Indicates the sequence of basis sets employed for the correlation energy.
\item[] \pyoptionname{DELTA\_BASIS} = \pyoptionval{\qq{cc-pV[TQ]Z}\qq} \textbar\; \pyoptionval{\qq{jun-cc-pv[tq5]z}\qq} \textbar\; \pyoptionval{\qq{6-31G}\qq} \textbar\; etc. \\
Indicates the sequence of basis sets employed for the delta correction to the correlation energy.
\item[] \pyoptionname{DELTA2\_BASIS} = \pyoptionval{\qq{cc-pV[TQ]Z}\qq} \textbar\; \pyoptionval{\qq{jun-cc-pv[tq5]z}\qq} \textbar\; \pyoptionval{\qq{6-31G}\qq} \textbar\; etc. \\
Indicates the sequence of basis sets employed for second delta correction to the correlation energy.
\end{itemize}

\item Schemes

Transformations of the energy through basis set extrapolation for each stage of the CBS definition.
A complaint is generated if number of basis sets in \pyoptionname{stage\_basis} does not exactly satisfy
requirements of \pyoptionname{stage\_scheme}.
An exception is the default, \pyoptionval{\qq{highest\_1}\qq}, which uses the best basis set available.
\begin{itemize}
\item[] \pyoptionname{SCF\_SCHEME} = \textit{\pyoptionval{\qq{highest\_1}\qq}} \textbar\; \pyoptionval{\qq{scf\_xtpl\_helgaker\_3}\qq} \\
Indicates the basis set extrapolation scheme to be applied to the reference energy.
\item[] \pyoptionname{CORL\_SCHEME} = \textit{\pyoptionval{\qq{highest\_1}\qq}} \textbar\; \pyoptionval{\qq{corl\_xtpl\_helgaker\_2}\qq} \\
Indicates the basis set extrapolation scheme to be applied to the correlation energy.
\item[] \pyoptionname{DELTA\_SCHEME} = \textit{\pyoptionval{\qq{highest\_1}\qq}} \textbar\; \pyoptionval{\qq{corl\_xtpl\_helgaker\_2}\qq} \\
Indicates the basis set extrapolation scheme to be applied to the delta correction to the correlation energy.
\item[] \pyoptionname{DELTA2\_SCHEME} = \textit{\pyoptionval{\qq{highest\_1}\qq}} \textbar\; \pyoptionval{\qq{corl\_xtpl\_helgaker\_2}\qq} \\
Indicates the basis set extrapolation scheme to be applied to the second delta correction to the correlation energy.
\end{itemize}
\end{itemize}

\hiddensubsubsection{Examples}

\begin{Snippet}
# DT-zeta extrapolated mp2 correlation energy atop a T-zeta reference
cbs('mp2', corl_basis='cc-pv[dt]z', corl_scheme=corl_xtpl_helgaker_2)

# a DT-zeta extrapolated coupled-cluster correction atop a TQ-zeta extrapolated
#   mp2 correlation energy atop a Q-zeta reference
cbs('mp2', corl_basis='aug-cc-pv[tq]z', corl_scheme=corl_xtpl_helgaker_2, \
    delta_wfn='ccsd(t)', delta_basis='aug-cc-pv[dt]z', delta_scheme=corl_xtpl_helgaker_2)

# cbs() coupled with database()
database('mp2', 'BASIC', subset=['h2o','nh3'], symm='on', func=cbs, \
    corl_basis='cc-pV[tq]z', corl_scheme=corl_xtpl_helgaker_2, \
    delta_wfn='ccsd(t)', delta_basis='sto-3g')

\end{Snippet}

\hiddensubsubsection{Output}

At the beginning of a \texttt{cbs()} job is printed a listing of the individual energy
calculations which will be performed. The output snippet below is from the second example
job above. It shows first each model chemistries
needed to compute the aggregate model chemistry requested through \texttt{cbs()}. Then,
since, for example, an \texttt{energy(\qq{ccsd(t)}\qq)} yields CCSD(T), CCSD, MP2, and SCF
energy values, the wrapper condenses this task into the second list of minimum number of
calculations which will actually be run.

\begin{Snippet}
    Naive listing of computations required.
            scf / aug-cc-pvqz              for  SCF TOTAL ENERGY
            mp2 / aug-cc-pvtz              for  MP2 CORRELATION ENERGY
            mp2 / aug-cc-pvqz              for  MP2 CORRELATION ENERGY
        ccsd(t) / aug-cc-pvdz              for  CCSD(T) CORRELATION ENERGY
        ccsd(t) / aug-cc-pvtz              for  CCSD(T) CORRELATION ENERGY
            mp2 / aug-cc-pvdz              for  MP2 CORRELATION ENERGY
            mp2 / aug-cc-pvtz              for  MP2 CORRELATION ENERGY

    Enlightened listing of computations required.
            mp2 / aug-cc-pvqz              for  MP2 CORRELATION ENERGY
        ccsd(t) / aug-cc-pvdz              for  CCSD(T) CORRELATION ENERGY
        ccsd(t) / aug-cc-pvtz              for  CCSD(T) CORRELATION ENERGY
\end{Snippet}

At the end of a \texttt{cbs()} job is printed a summary section like the one below. First,
in the components section, are listed the results for each model chemistry available, whether
required for the cbs job (*) or not. Next, in the stages section, are listed the results for
each extrapolation. The energies of this section must be dotted with the weightings in column Wt
to get the total cbs energy. Finally, in the CBS section, are listed the results for each stage
of the cbs procedure. The stage energies of this section sum outright to the total cbs energy.

\begin{Snippet}
 ==> Components <==

-------------------------------------------------------------------------------------
                  Method / Basis            Rqd   Energy [H]   Variable
-------------------------------------------------------------------------------------
                     scf / aug-cc-pvqz        *  -1.11916375   SCF TOTAL ENERGY
                     mp2 / aug-cc-pvqz        *  -0.03407997   MP2 CORRELATION ENERGY
                     scf / aug-cc-pvdz           -1.11662884   SCF TOTAL ENERGY
                     mp2 / aug-cc-pvdz        *  -0.02881480   MP2 CORRELATION ENERGY
                 ccsd(t) / aug-cc-pvdz        *  -0.03893812   CCSD(T) CORRELATION ENERGY
                    ccsd / aug-cc-pvdz           -0.03893812   CCSD CORRELATION ENERGY
                     scf / aug-cc-pvtz           -1.11881134   SCF TOTAL ENERGY
                     mp2 / aug-cc-pvtz        *  -0.03288936   MP2 CORRELATION ENERGY
                 ccsd(t) / aug-cc-pvtz        *  -0.04201004   CCSD(T) CORRELATION ENERGY
                    ccsd / aug-cc-pvtz           -0.04201004   CCSD CORRELATION ENERGY
-------------------------------------------------------------------------------------

 ==> Stages <==

-------------------------------------------------------------------------------------
    Stage         Method / Basis             Wt   Energy [H]   Scheme
-------------------------------------------------------------------------------------
      scf            scf / aug-cc-pvqz        1  -1.11916375   highest_1
     corl            mp2 / aug-cc-pv[tq]z     1  -0.03494879   corl_xtpl_helgaker_2
    delta        ccsd(t) / aug-cc-pv[dt]z     1  -0.04330347   corl_xtpl_helgaker_2
    delta            mp2 / aug-cc-pv[dt]z    -1  -0.03460497   corl_xtpl_helgaker_2
-------------------------------------------------------------------------------------

 ==> CBS <==

-------------------------------------------------------------------------------------
    Stage         Method / Basis                  Energy [H]   Scheme
-------------------------------------------------------------------------------------
      scf            scf / aug-cc-pvqz           -1.11916375   highest_1
     corl            mp2 / aug-cc-pv[tq]z        -0.03494879   corl_xtpl_helgaker_2
    delta  ccsd(t) - mp2 / aug-cc-pv[dt]z        -0.00869851   corl_xtpl_helgaker_2
    total            CBS                         -1.16281105
-------------------------------------------------------------------------------------
\end{Snippet}

\hiddensubsubsection{Extrapolation Schemes}

\begin{itemize}

\item \pyoptionval{highest\_1}

   \begin{itemize}
   \item[] Suitable Methods: all
   \item[] Suitable Basis Sets: single basis or highest $\zeta$ among array of bases
   \item[] Governing Equation: $E_{total}^{X} = E_{total}^{X}$
   \item[] Optional Arguments: none
   \end{itemize}

\item \pyoptionval{scf\_xtpl\_helgaker\_3}

   \begin{itemize}
   \item[] Suitable Methods: reference
   \item[] Suitable Basis Sets: three adjacent $\zeta$-level bases
   \item[] Governing Equation: $E_{total}^{X} = E_{total}^{\infty} + \beta e^{-\alpha X}$
   \item[] Optional Arguments: none
   \end{itemize}

\item \pyoptionval{scf\_xtpl\_helgaker\_2}

   \begin{itemize}
   \item[] Suitable Methods: reference
   \item[] Suitable Basis Sets: two adjacent $\zeta$-level bases
   \item[] Governing Equation: $E_{total}^{X} = E_{total}^{\infty} + \beta e^{-\alpha X}$
   \item[] Optional Arguments: none (someday will be allowed to set $\alpha$ through argument)
   \end{itemize}

\item \pyoptionval{corl\_xtpl\_helgaker\_2}\cite{Halkier:1998:CBS}

   \begin{itemize}
   \item[] Suitable Methods: correlated
   \item[] Suitable Basis Sets: two adjacent $\zeta$-level bases
   \item[] Governing Equation: $E_{corl}^{X} = E_{corl}^{\infty} + \beta X^{-3}$
   \item[] Optional Arguments: none
   \end{itemize}

\end{itemize}

%\item \texttt{scf\_xtpl\_helgaker\_2}
%
%   \begin{itemize}
%   \item[] Suitable Methods:
%   \item[] Suitable Basis Sets:
%   \item[] Governing Equation:
%   \item[] Optional Arguments:
%   \end{itemize}


\subsection{Compound Wrapper Calls}\label{wrapcompound}

The wrappers below can be called in combination according to Table \ref{table:wrapintercalls}
below, so that db(opt(cbs(energy())) is permitted while
db(cp(energy())) is not. Note also that the command
db(opt(cbs(energy())) is actually expressed
as \texttt{db($\dots$, db\_func=opt, opt\_func=cbs)}. The perhaps expected final
argument of \texttt{cbs\_func=energy} is not necessary since the \texttt{energy()}
is always the function called by default. Also, the outermost internal function call
(\texttt{db\_func} above) can be called as just \texttt{func}. Several examples of intercalls
between wrappers can be found in sample input \file{pywrap\_all}.
Table not yet validated for calls with cp().

\begin{table}[!htbp]
\begin{footnotesize}
\caption{Permitted intercalls between python wrappers.} \label{table:wrapintercalls}
\parsep 10pt
\begin{center}
\begin{tabular}{lccccc}
\hline\hline
Caller & \multicolumn{5}{c}{Callee} \\
\cline{2-6}
& \texttt{cp} & \texttt{db} & \texttt{opt} & \texttt{cbs} & \texttt{energy} \\
\hline
\texttt{cp}     &    & -- & Y  & Y  & Y \\
\texttt{db}     & -- &    & Y  & Y  & Y \\
\texttt{opt}    & -- & -- &    & Y  & Y \\
\texttt{cbs}    & -- & -- & -- &    & Y \\
\texttt{energy} & -- & -- & -- & -- &   \\
\hline\hline
\end{tabular}
\end{center}
\end{footnotesize}
\end{table}

