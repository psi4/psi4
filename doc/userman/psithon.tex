\section{Psithon} \label{psithon} 
To allow arbitrarily complex computations to be performed, \PSIfour was built
upon the Python interpreter.  However, to make the input syntax simpler, some
pre-processing of the input file is performed before it is interpreted,
resulting in Python syntax that is customized for \PSI, termed Psithon.

\subsection{Molecule Specification}

\PSIfour has a very flexible input parser that allows many different methods of
molecule specification.  

\subsection{Job Control}

\PSI comprises a number of modules, written in C++, that each perform specific
tasks and are callable directly from the Python front end.  Each module
recognizes specific keywords in the input file, detailed in chapter
\ref{keywords}, which control its function.  For example, consider the
following The keywords can be made global, or scoped to apply to certain
subsets of the input file.

\begin{Snippet}
molecule {
  o
  h 1 roh
  h 1 roh 2 ahoh

  roh = 0.957
  ahoh = 104.5
}

set basis cc-pVDZ
set ccenergy print 3
set scf print 1

energy('ccsd')
\end{Snippet}

To perform a CCSD computation, an SCF computation is performed first, before
the \PSIccenergy module is called to perform the coupled cluster computation;
the {\tt energy} Python helper function ensures that this is performed
correctly.  The {\tt set} keyword is one of the custom \PSI keywords and it is
used to provide job control keywords.  In the above example, the basis keyword
is set to ``cc-pVDZ'' throughout the code.  The following equivalent statements
can be used to set ``global'' options, {\it i.e.} those that are applied
throughout the remainder of the input file.

\begin{Snippet}
set globals basis cc-pVDZ
set basis cc-pVDZ
set globals basis = cc-pVDZ
set basis = cc-pVDZ
\end{Snippet}

Notice the lack of quotes, despite
the value being a string.  The Psithon preprocessor identifies the {\tt set}
keywords and wraps any strings in quotes automatically.  On lines 11 and 12, we
demonstrate the scoping of variables.  Line 11 sets the print level for the
coupled cluster module to 3, while line 12 sets the SCF module's print level to
1.

The above input file does the exact same thing that the input file provided
in chapter \ref{tutorial}'s {\em Running a basic SCF calculation}. It provides
a simple single-point energy computation on water.

Both of these input files are automatically preprocessed to Python for you.

\subsection{Preprocessor Keywords}

\begin{description}
\item[molecule {\em name } { ... }]\mbox{}\\
Defines a molecule.

\end{description}
