%
% The PSI Installation Manual
%

\documentclass[12pt]{article}
\usepackage{hyperref}
\usepackage{fancyvrb}
\usepackage{xspace}
\setlength{\textheight}{9in}
\setlength{\textwidth}{6.5in}
\setlength{\hoffset}{0in}
\setlength{\voffset}{0in}
\setlength{\headheight}{0in}
\setlength{\headsep}{0in}
\setlength{\topmargin}{0in}
\setlength{\oddsidemargin}{-0.05in}
\setlength{\evensidemargin}{-0.05in}
\setlength{\marginparsep}{0in}
\setlength{\marginparwidth}{0in}
\setlength{\parsep}{0.8ex}
\setlength{\parskip}{1ex plus \fill}
\baselineskip 18pt
\renewcommand{\topfraction}{.8}
\renewcommand{\bottomfraction}{.2}

\begin{document}
\include{macros}

\begin{center}
\ \\
\vspace{2.0in}
{\bf {\Large Installation Manual for the \PSIfour\ Program Package}} \\
\vspace{0.5in}
T.\ Daniel Crawford$^a$, C.\ David Sherrill$^b$, and
Edward F.\ Valeev$^a$
\\ \  \\
{\em $^a$Department of Chemistry, Virginia Tech, Blacksburg, 
Virginia 24061-0001} \\
\vspace{0.1in}
{\em $^b$Center for Computational Molecular Science and Technology, 
\mbox{Georgia Institute of Technology,} Atlanta, Georgia 30332-0400} \\
\vspace{0.1in}
\ \\
\vspace{0.3in}
\PSIfour\ Version: \PSIversion \\
Created on: \today
\end{center}

\thispagestyle{empty}

\newpage
\section{Compilation Prerequisites}

The following external software packages are needed to complile \PSIfour:
\begin{itemize}
\item C++ Compiler
\item F77 compiler (the F95 compiler, gfortran, with gcc-4.X will work);
  this is only used to determine the symbol-naming
  convention of and some system routines for the BLAS and LAPACK libraries
  on a few architectures.  It is optional in a few cases (e.g. Mac OS X
  systems)
\item A well-optimized basic linear algebra subroutine (BLAS) library
  for vital matrix-matrix and matrix-vector multiplication
  routines. (See recommendations below.)
\item The linear algebra package (LAPACK) (see recommendations below)
\item POSIX threads (Pthreads) library
\item Perl interpreter (version 5.005 or higher)
\item Python interpreter (2.4 or higher, but not version 3.x; 
      needed for input files)
\item Python developer libraries corresponding to your interpreter
\item A version of MPI is required to compile distributed-parallel PSI; 
  MPICH2 is recommended.
  (Note: MPICH2-1.1.0 had trouble with some header files.  MPICH2-1.2.1
  seems to fix it.)
\item Various GNU utilies: \htmladdnormallink{{\tt
www.gnu.org}}{http://www.gnu.org}
\begin{itemize}
\item {\tt make}
\item {\tt autoconf (version 2.52 or higher)}
\item {\tt aclocal}
\item {\tt fileutils} (esp.\ {\tt install})
\end{itemize}
\item For documentation only:
\begin{itemize}
\item {\tt LaTeX}
%\item {\tt LaTeX2html} (v0.99.1 or 1.62, including the patch supplied in
%psi3/misc)
\end{itemize}
\item To make a distributed-parallel version with MADNESS, you may also
need 
\begin{itemize}
\item libtool (called by autoreconf)
\end{itemize}
\end{itemize}

   To check to see if you have the Python developer libraries installed look
   for the Python config program. If you Python interpreter is named `python'
   look for the config program `python-config', likewise if your interpreter
   is named `python2.6' your config program is named `python2.6-config'. If
   you cannot find the config program the developer libraries will not be
   detected and the PSI4 configure script will fail.

   For Ubuntu users, you will need the following packages installed:
   gfortran [for linking to BLAS/LAPACK], g++, autoconf, python-dev

   For Fedora users, the python development library is called python-devel

\section{Brief Summary of Configuration, Compilation, and Installation}

This section outlines the main steps of configuring, compiling, and
installing PSI.  More detail is given below.

\subsection{Autoconf}
\subsubsection{In the top-level psi4 directory, run autoconf:}
\begin{enumerate}
\item {\tt autoconf}
\end{enumerate}

\subsubsection{Distributed-parallel compilation}

Not recommended at this time except by developers.  Shared-memory
parallelization is already enabled by default in the standard
compilation.

Distributed-parallel versions of PSI4 require {\tt madness}. If you select
mpicxx as the compiler, the distributed-parallel version (including
{\tt madness}) will compile. For distributed-parallel compilation, you must
run the following command in the {\tt madness} directory, otherwise the PSI4
configure script will fail (autoreconf is provided by package autoconf,
but it calls another program provided by libtool, so that package must
also be installed):

\begin{enumerate}
\item {\tt cd madness}
\item {\tt autoreconf}
\item {\tt cd ..}
\end{enumerate}

\subsection{Configuration and Compilation}

Make an object directory in which you can compile the code
\begin{enumerate}
\item {\tt mkdir obj}
\end{enumerate}

\noindent
Next, you need to configure the code, make it, and run the tests.

\noindent
Either

\begin{enumerate}
\item {\tt cd obj}
\item run ../configure with one of the options below
\item {\tt make}
\item {\tt make tests}
\item {\tt make install} (if tests pass)
\item {\tt make doc} (optional)
\item do user configuration, see below
\end{enumerate}

\noindent
or, if you want to save your configuration options for a future compilation,
then in the top-level psi4 directory,
\begin{enumerate}
\item create a file like ``do-configure'' with your configuration options,
      e.g., as below
\item {\tt chmod u+x do-configure}
\item {\tt cd obj}
\item {\tt ../do-configure}
\item {\tt make}
\item {\tt make tests}
\item {\tt make install} (if tests pass)
\item {\tt make doc} (optional)
\item do user configuration, see below
\end{enumerate}

That's it!  The details about configuration are given below.  If something
goes wrong, see below about common compilation problems.

\section{Detailed Installation Instructions}

This section provides a more detailed explanation of the procedure for
compiling and installing the PSI4 package.

\subsection{Configuration}

\subsubsection{General Information about Configuration}

First, we recommend that you choose for the top-level {\tt \$PSI4} source
directory something other than {\tt /usr/local/psi}; {\tt \$HOME/psi4}
or {\tt /usr/local/src/psi4} are convenient choices.  Next, in the
top-level {\tt \$PSI4} source directory you've chosen, first run autoconf
to generate the {\tt configure} script from {\tt configure.ac}.  It is
best to keep the source code separate from the compilation area, so you
must first choose a subdirectory for compilation of the codes.  A simple
option is {\tt \$PSI4/objdir}, which should work for most environments.
However, if you need executables for several architectures, you should
choose more meaningful subdirectory names.

{\em The compilation directory will be referred to as \$objdir for the
   remainder of these instructions.
}

In \$objdir, run the configure script found in the PSI4 top-level
source directory.  This script will scan your system to locate
certain libraries, header files, etc. needed for complete compilation.
The script accepts a number of options, all of which are listed above.
The most important of these is the {\tt --prefix} option, which selects
the installation directory for the executables, the libraries, header
files, basis set data, and other administrative files.  The default
{\tt --prefix} is {\tt /usr/local/psi}.

Besides {\tt --prefix}, PSI often needs a few additional options for
the configure script.  To make it reasy to recompile later (especially
if you're a developer), it can be convenient (but not necessary) to
to put the configure options in a small executable file, so you can
re-do the configuration later very easily. Let us assume that we will
be putting the configure options in a file named {\tt do-configure},
in the top-level psi4 directory (we'll keep it up there instead of down
in the compilation directory {\tt \$objdir}, so that if we delete the
compilation directory later, we'll still have the {\tt do-configure}
file). In the following examples, we have sometimes split the configure
options over multiple lines for readability. However, they should all
be on the same line in the do-configure script.

{\em Note: The configure options below are for the most common
architectures and compilers. The developers would appreciate it if you
would share any special configuration options that might be needed for
less commonly encountered situations.  }

For g++, if you have BLAS and LAPACK in standard locations (like {\tt
/usr/lib64}), configuration is very easy. Pick one of the following
scenarios, and place the text given in the {\tt psi4/do-configure} file
(all on one long line). Replace the text after prefix with whatever
directory you want to use for your installation directory.

\noindent
{\bf g++, optimized} \\
{\tt ../configure --prefix=/usr/local/psi4}

\noindent
{\bf g++, for debugging} \\
{\tt ../configure --prefix=/usr/local/psi4 --without-opt --with-debug}

\pagebreak

\noindent
{\bf Intel compiler with MKL math library} \\
{\tt ../configure --prefix=/usr/local/psi4 --with-blas='-mkl' --with-cc=icc
--with-cxx=icpc --with-fc=ifort  --with-opt='-O2 -static -no-prec-div'
--with-incdirs=-mkl}

\noindent
{\bf Compiling for Mac} \\
To get the compilers needed, it's easiest to install XCode. However, XCode
does not provide a Fortran compiler. Although Fortran compilers are not needed
to compile Psi, a broken one can prevent Psi from configuring properly. Do not
download the latest version of GFortran from the HPC website; this is unlikely
to be compatible with your version of GCC. Instead, you should run {\tt gcc
-v} to find out what version of GCC you're using, and then download the
corresponding GFortran from  here. If you configure Psi on a Mac without any
Fortran compiler it will set itself up correctly, so this is only necessary if
you want a Fortran compiler for other purposes. You can configure Psi by
adding something like
\begin{quotation}
\noindent
{\tt ../configure --with-plugins}
\end{quotation}
to the do-configure script. If you want to use the new LLVM compilers that
ship with XCode 4 (they compile quicker than GCC), use
\begin{quotation}
\noindent
{\tt ../configure --with-plugins --with-cxx=llvm-g++}
\end{quotation}

If you still happen to encouter an error like:
\begin{quotation}
\noindent
{\tt checking Fortran symbols... giving up} \\
{\tt configure: error: could not determine fortran symbol names} 
\end{quotation}
adding the following tag to your configure may help:
\begin{quotation}
{\tt --with-f77symbol=lcu}
\end{quotation}

\subsubsection{List of Specific Configuration Options}

You may need to make use of one or more of the following options to
the {\tt configure} script:
\begin{itemize}
\item {\tt -}{\tt -prefix=directory} --- Use this option if you wish to
  install the \PSIfour\ package somewhere other than the default
  directory, {\tt /usr/local/psi}. 
\item {\tt -}{\tt -with-cxx=compiler} --- Use this option to specify a
  C++ compiler.  One should use compilers that generate reentrant
  code, if possible. The default search order for compilers is: {\tt xlC\_r}
  (AIX only), {\tt g++}, {\tt c++}, {\tt icpc}, {\tt cxx}.
  For distributed-parallel compilation, MPI is required and you need to use
  mpicxx (where this has been added to your PATH).
\item {\tt -}{\tt -with-fc=compiler} --- Use this option to specify a
  Fortran-77 compiler, which is used to determine linking coventions
  for BLAS and LAPACK libraries and to provide system routines for
  those libraries.  Note that no fortran compiler is necessary on Mac
  OS X systems (see below).  The default search order for compilers
  is: {\tt xlf\_r} (AIX only), {\tt gfortran}, {\tt g77}, {\tt ifort},
  {\tt f77}, {\tt f2c}.
\item {\tt -}{\tt -with-f77-symbol=value} --- This option allows manual
  assignment of the F77 symbol convention, which is necessary for C
  programs to link Fortran-interface libraries such as BLAS and
  LAPACK. This option should only be used by experts and even then
  should almost never be necessary.  Allowed values are:
\begin{itemize}                            
\item[lc] lower-case
\item[lcu]lower-case with underscore (default)
\item[uc] upper-case
\item[ucu] upper-case with underscore
\end{itemize}
\item {\tt -}{\tt -with-ld=linker} --- Use this option to specify
  a linker program. The default is {\tt ld}.
\item {\tt -}{\tt -with-ranlib=ranlib} --- Use this option to specify
  a ranlib program. The default behavior is to detect an appropriate
  choice automatically.
\item {\tt -}{\tt -with-ar=archiver} --- Use this option to specify an
  archiver.  The default is to look for {\tt ar} automatically.
\item {\tt -}{\tt -with-ar-flags=options} --- Use this option to specify
  archiver command-line flags. The default is {\tt r}.
\item {\tt -}{\tt -with-incdirs=directories} --- Use this option to
  specify extra directories where to look for header
  files. Directories should be specified prepended by {\tt -I},
  i.e. {\tt -Idir1 -Idir2}, etc. If several directories are specified,
  enclose the list with single right-quotes, e.g., {\tt -}{\tt
    -with-incdirs='-I/usr/local/include -I/home/psi3/include'}.
\item {\tt -}{\tt -with-libs=libraries} --- Use this option to specify
  extra libraries which should be used during linking. Libraries
  should be specified by their full names or in the usual {\tt -l}
  notation, i.e. {\tt -lm /usr/lib/libm.a}, etc.  If several libraries
  are specified, enclose the list with single right-quotes, e.g., {\tt
    -}{\tt -with-libs='-lcompat /usr/local/lib/libm.a'}.
\item {\tt -}{\tt -with-libdirs=directories} --- Use this option to
  specify extra directories where to look for libraries. Directories
  should be specified prepended by {\tt -L}, i.e. {\tt -Ldir1 -Ldir2},
  etc. If several directories are specified, enclose the list with
  single right-quotes, e.g., {\tt -}{\tt
    -with-libdirs='-L/usr/local/lib -I/home/psi3/lib'}.
\item {\tt -}{\tt -with-blas=library} --- Use this option to specify a
  BLAS library.  If your BLAS library has multiple components, enclose
  the file list with single right-quotes, e.g., {\tt -}{\tt
    -with-blas='-lf77blas -latlas'}.  Note that many BLAS libraries
  can be detected automatically.
\item {\tt -}{\tt -with-lapack=library} --- Use this option to specify
  a LAPACK library.  If your LAPACK library has multiple components,
  enclose the file list with single right-quotes, e.g., {\tt -}{\tt
    -with-lapack='-llapack -lcblas -latlas'}.  note that many LAPACK
  libraries can be detected automatically.
\item {\tt -}{\tt -with-max-am-eri=integer} --- Specifies the maximum
  angular momentum level for the primitive Gaussian basis functions
  when computing electron repulsion integrals.  This is set to
  $g$-type functions (AM=4) by default.
\item {\tt -}{\tt -with-max-am-deriv1=integer} --- Specifies the maximum
  angular momentum level for first derivatives of the primitive
  Gaussian basis functions.  This is set to $f$-type functions (AM=3)
  by default.
\item {\tt -}{\tt -with-max-am-deriv2=integer} --- Specifies the maximum
  angular momentum level for second derivatives of the primitive
  Gaussian basis functions.  This is set to $d$-type functions (AM=2)
  by default.
\item {\tt -}{\tt -with-max-am-r12=integer} --- Specifies the maximum
  angular momentum level for primitive Gaussian basis functions used
  in $r_{12}$ explicitly correlated methods.  This is set to $f$-type
  functions (AM=3) by default.
\item {\tt -}{\tt -with-debug=yes/no} --- This option turns on debugging
  options.  This is set to {\tt no} by default.
\item {\tt -}{\tt -with-opt=options} --- Turn off compiler
  optimizations if {\tt no}.  This is set to {\tt yes} by default.
\item {\tt -}{\tt --with-strict=yes} -- Turns on strict compiler warnings.
\end{itemize}

\noindent
{\bf Python Interpreter:} \\
   Usually Python will be detected automatically.  If this fails, or if
   you have multiple versions installed and want to specify a particular
   one, set the {\tt PYTHON} environmental variable to the full path name
   of the Python interpreter you want to use.  This defaults to the
   {\tt `python'} in your path. For example, if you want to use
   {\tt `python2.6'} located in /usr/bin set the environmental variable to be:

\begin{quotation}
\noindent
      {\tt PYTHON=/usr/bin/python2.6}
\end{quotation}

   Please note if set the variable the config program must be present with
   a similar name. For instance, in the above example the following must
   exist:

\begin{quotation}
\noindent
      {\tt /usr/bin/python2.6-config}
\end{quotation}

   You either set the environmental variable before you call configure, or
   tell configure about it:

\begin{quotation}
\noindent
      {\tt ./configure PYTHON=/usr/bin/python2.6}
\end{quotation}

\subsection{Step 2: Compilation}

Running {\tt make} (which must be GNU's {\tt 'make'} utility) in {\tt
\$objdir} will compile the \PSIfour\ libraries and executable
modules.

\subsection{Step 3: Testing}

To execute automatically the ever-growing number of test cases after
compilation, simply execute ``make tests'' in the {\tt \$objdir}
directory.  This will run each (relatively small) test case and report
the results.  Failure of any of the test cases should be reported to
the developers at \PSIemail. By default, any such failure will stop
the testing process.  If you desire to run the entire testing suit
without interruption, execute ``make tests TESTFLAGS='-u -q' ''. Note
that you must do a ``make testsclean'' in {\tt \$objdir} to run the test
suite again.

\subsection{Step 4: Installation}

Once testing is complete, installation into \$prefix is accomplished by
running {\tt make install} in {\tt \$objdir}.   Executable modules are
installed in {\tt \$prefix/bin}, libraries in {\tt \$prefix/lib} and basis 
set data and other control strctures {\tt \$prefix/share}.

\subsection{Step 5: Documentation}

If your system has the appropriate utilities, you may build the package
documentation from the top-level {\tt \$objdir} by running {\tt make doc}.  
The resulting files will appear in the {\tt \$prefix/doc} area.

\subsection{Step 6: Cleaning}

All compilation-area object files and libraries can be removed to save
disk space by running {\tt make clean} in {\tt \$objdir}.

\subsection{Step 7: User Configuration}

After the \PSIfour\ package has been successfullly installed, the user will
need to add the installation directory into their path.  If the package
has been installed in the default location {\tt /usr/local/psi}, then
in C shell, the user should add something like the following to 
their {\tt .cshrc} file:
\begin{quotation}
\noindent
{\tt setenv PSI /usr/local/psi} \\
{\tt set path = (\$path \$PSI/bin)}
\end{quotation}

Next, you need to create a file {\tt .psi4rc} in each user's home
directory.  A sample file is provided in the PSI4 source ({\tt
psi4/samples/example\_psi4rc\_file}).  This file tells the PSI4 I/O
manager how to handle scratch files.

Copy this file to each user's home directory

\begin{quotation}
{\tt cp psi4/samples/example\_psi4rc\_file \$HOME/.psi4rc}
\end{quotation}

Now edit this file.  Replace the path in the following line

\begin{quotation}
{\tt psioh.set\_default\_path('/scratch/parrish/')}
\end{quotation}
with a path to a fast scratch disk for which the user has write access.  Note:
This should NOT be an NFS-mounted volume, as writes to a remote disk over the
network can be very slow and can tie up the network and negatively impact
other users.  If the local /tmp volume is large enough, it might be used.
However, a dedicated scratch volume (using RAID0 striping for speed) is
recommended.

For developers: during compilation and testing, PSI4 finds its
basis sets, grids, etc., in {\tt psi4/lib}.  After installation,
PSI4 will look in {\tt \$prefix/share/psi}.  If you want to specify a
non-standard location for this information, you can do this by setting
the environmental variable {\tt \$PSI4DATADIR} to the directory containg
the basis, grids, etc., subdirectories.


\section{Recommendations for BLAS and LAPACK Libraries}

Much of the speed and efficiency of the PSI4 programs depends on the
corresponding speed and efficiency of the available BLAS and LAPACK
libraries (especially the former).  In addition, the most common
compilation problems involve these libraries.  Users may therefore
wish to consider the following BLAS and LAPACK recommendations when
building PSI4:

\begin{itemize}
\item It is NOT wise to use the stock BLAS library provided with many
  Linux distributions like RedHat.  This library is usually just the
  netlib ({http://netlib.org/}distribution and is completely
  unoptimized.  PSI4's performance will suffer if you choose this
  route.  The choice of LAPACK is less critical, and so the
  unoptimized netlib distribution is acceptable.  If you do choose to
  use the RedHat/Fedora stock BLAS and LAPACK, be aware that some
  RPM's do not make the correct symbolic links.  For example, you may
  have {\tt /usr/lib/libblas.so.3.1.0} but not {\tt
    /usr/lib/libblas.so}.  If this happens, create the link as, e.g.,
  {\tt ln -s /usr/lib/libblas.so.3.1.0 /usr/lib/libblas.so}.  You may
  need to do similarly for lapack.

\item Perhaps the best choices for BLAS are Kazushige Goto's
  hand-optimized BLAS \\
({\tt http://www.tacc.utexas.edu/resources/software/}) and ATLAS \\ ({\tt
http://math-atlas.sourceforge.net/}).  These work well on nearly
  every achitecture to which the PSI4 developers have access, though we have
  identified at least one case in which the Goto libraries yielded faulty
  DGEMM call.  On Mac OS X systems, however, the {\tt vecLib} package that
  comes with Xcode works well. Note also that we have encountered problems
  with version 10 of Intel's MKL, particularly for very large coupled
  cluster calculations.

\item PSI4 does not require a Fortran compiler, unless the resident
  BLAS and LAPACK libraries require Fortran-based system libraries.
  If you see compiler complaints about missing symbols like ``{\tt
    do\_fio}'' or ``{\tt e\_wsfe}'' then your libraries were most likely
  compiled with g77 or gfortran, which require {\tt -lg2c} to resolve
  the Fortran I/O calls.  Use of the same gcc package for PSI4 should
  normally resolve this problem.

\item The PSI4 configure script can often identify and use
  several different BLAS and LAPACK libraries, but its ability to do
  this automatically depends on a number of factors, including
  correspondence between the compiler used for PSI4 and the compiler
  used to build BLAS/LAPACK, and placement of the libraries in
  commonly searched directories, among others.  PSI4's configure
  script will find your BLAS and LAPACK if any of the the following
  are installed in standard locations (e.g. {\tt /usr/local/lib}):

\begin{itemize}  
    \item ATLAS: {\tt libf77blas.a} and {\tt libatlas.a}, plus netlib's
    {\tt liblapack.a}
    \item MKL 8: {\tt libmkl.so} and {\tt libmkl\_lapack64.a} (with the corresponding Intel compilers)
    \item Goto: {\tt libgoto.a} and netlib's {\tt liblapack.a}
    \item Cray SCSL (e.g. on SGI Altix): {\tt libscs.so} (NB: No Fortran compiler
      is necessary in this case, so {\tt -}{\tt -with-fc=no} should work.)
    \item ESSL (e.g. on AIX systems): {\tt libessl.a}
    \end{itemize}  
  \item If configure cannot identify your BLAS and LAPACK libraries
    automatically, you can specify them on the command-line using the
    {\tt -}{\tt -with-blas} and {\tt -}{\tt -with-lapack} arguments
    described above.  Here are a few examples that work on the PSI4
    developers' systems:
  
    (a) Linux with ATLAS:
  
    {\tt -}{\tt -with-blas='-lf77blas -latlas'} {\tt -}{\tt -with-lapack='-llapack -lcblas'}

    (b) Mac OS X with vecLib: 
  
    {\tt -}{\tt -with-blas='-altivec -framework vecLib'} {\tt -}{\tt -with-lapack=' '}
  
    (c) Linux with MKL 8.1 and {\tt icc/icpc/ifort} 9.1: 
  
    {\tt -}{\tt -with-libdirs=-L/usr/local/opt/intel/mkl/8.0.2/lib/32} {\tt -}{\tt -with-blas=-lmkl} {\tt -}{\tt -with-lapack=-lmkl\_lapack32}
\end{itemize}

    (d) Linux on ia32 with MKL 10.1 and {\tt icc/icpc} 11.0:

    {\tt -}{\tt -with-blas='-Wl,--start-group -L/usr/local/opt/intel/mkl/10.1.0.015/lib/32 -lmkl -Wl,}{\tt -}{\tt -end-group -lguide -lpthread'}

\subsection{Compilation notes for ATLAS}
   These shortcut notes might be helpful if you are using Linux.  However,
   we recommend reading and following the full ATLAS installation notes.

   \begin{itemize}
   
   \item
   You'll need a Fortran compiler installed.

   \item
   Unpack the source code, then make a compilation directory (could
   be an {\tt obj} subdirectory in the source directory, or elsewhere).

   \item
   Turn off CPU throttling so the auto-tuning capabilities have a chance
   to work.  On Linux, this can be tune using

     {\tt /usr/bin/cpufreq-selector -g performance}

   \item
   {\tt cd} into the compilation directory and run the source
   directory configure script there, with any necessary flags, e.g.,
 
     {\tt /usr/local/src/atlas/configure --prefix=/usr/local/atlas}

   where {\tt prefix} gives the installation directory.
   It should automatically detect if you're on an {\tt x86\_64} 
   architecture and use 64-bit addressing ({\tt -b 64} flag) if so.

   \item
   Then make and check using

     {\tt make; make check; make ptcheck}

   \item
   And install

     {\tt make install}
   \end{itemize}

\subsection{Compilation notes for Netlib's LAPACK}
   These shortcut notes might be helpful if you are using Linux.  However,
   we recommend reading and following the full LAPACK installation notes.

   \begin{itemize}

   \item
   You'll need a Fortran compiler installed.

   \item
   If you decide to compile LAPACK from source, it may be obtained from \\
   {\tt http://www.netlib.org/lapack/}.  Unpack the source code, and in the
   top-level source directory, you need to create a make.inc file with
   the appropriate options for your machine.  For Linux/gfortran,
   simply

     {\tt cp make.inc.example make.inc}

   \item
   Next, edit {\tt BLASLIB} in {\tt make.inc} to point to your BLAS library
   (full pathnames are recommended), e.g.,

     {\tt
     BLASLIB      = /home/david/software/atlas3.9.25/lib/libf77blas.a \\
/home/david/software/atlas3.9.25/lib/libatlas.a
     }

   \item
   Edit Makefile as necessary (probably not needed).

   \item {\tt make}

   \item
   Copy the resulting file [{\tt lapack\_(\$ARCH).a}] where you want it
   (a standard location like {\tt /usr/local/lib} is easier for PSI to find).
   It is probably helpful to rename the file {\tt liblapack.a}.

   \end{itemize}


\section{Miscellaneous architecture-specific notes}
\begin{itemize}

\item Linux on x86 and x86\_64:
  \begin{itemize}
   \item Intel compilers: We had trouble with icpc 12.0.x.  Use 12.1 or later.
   \item Some versions of RedHat/Fedora Core RPM packages for the 
   BLAS and LAPACK libraries fail to make all the required symlinks.  
   For example, you may have {\tt /usr/lib/libblas.so.3.1.0} but not
   {\tt /usr/lib/libblas.so}.  If this happens, create the link as, e.g.,
   {\tt ln -s /usr/lib/libblas.so.3.1.0 /usr/lib/libblas.so}.  You
   may need to do something similar for lapack.
  \end{itemize}
\end{itemize}

\section{Common Problems with PSI Compilation}
\begin{itemize}

\item Compilation dies in the integrals code when compiling with the {\tt -j} 
flag

You can compile PSI with the {\tt -j} flag to compile with multiple
threads (e.g., {\tt -j2}). The integrals code is not safe for this and
will die. This is not a problem, just type 'make' again, and compilation
will resume. Once the integrals are compiled, this shouldn't be a
problem again.

\item {\tt No rule to make target foo.h, needed by bar.d. Stop}

This commonly happens after pulling updates from the repository. It happens
when a library header file is removed or renamed by the update, but there are
still old dependency files in the object directory, which think that they
still need to know about that header. There's a simple remedy, just run
\begin{quotation}
\noindent
{\tt  make DODEPEND=no dclean} 
\end{quotation}
in the object directory

\item Make gets stuck in an infinite loop

This means that the makefiles have not been properly updated. Running
\begin{quotation}
\noindent 
{\tt autoconf}
\end{quotation}
in the top-level Psi directory, followed by
\begin{quotation}
\noindent
{\tt ./config.status --recheck} \\
{\tt ./config.status} 
\end{quotation}
in the object directory should fix it. This procedure will need to be run
whenever an update changes the directory structure.

\item Incompatible g++/icpc

The Intel compilers require an installed set of C++ headers. Unfortunately,
the GNU compilers tend to be more cutting-edge than the Intel compilers,
meaning that Intel is always playing catch-up to new features in g++. This
means the two are often incompatible, leading to trouble if one wants to use
icpc to compile PSI4 (or anything else...). Your best bet in general is to not
upgrade Linux too fast, and always keep the very latest Intel compilers
around.

\item Missing symbols like {\tt do\_fio} or {\tt e\_wsfe}

See section on BLAS above.

\end{itemize}



\end{document}
